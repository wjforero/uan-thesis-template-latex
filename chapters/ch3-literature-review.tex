\chapter{Marco referencial}

En este marco, usted debe analizar todo lo que se ha escrito, desarrollado e
investigado sobre el objeto de su estudio.
Tenga en cuenta: lo que se sabe del tema, los estudios que se han realizado,
desde qué puntos se han tratado. Cuando se desarrolla el marco referencial o
teórico, se sustenta de forma teórica el problema de su proyecto, para ello, no
se limite únicamente a describirlo, sino que deberá revisar diferentes teorías y
aspectos de la misma. Procure que el objeto de su estudio sea esclarecido por
medio de las teorías que aborde, explique cómo y cuándo ocurre, establezca
relaciones internas y externas del fenómeno que investiga y defina de forma
clara cómo enfocará el problema para su estudio.

\section{Marco teórico}

En el marco teórico se reúne información documental con el fin de establecer
el cómo y qué información se acopie, de qué manera se analizará y estimar el
tiempo en que se demore el desarrollo.
La información seleccionada en el marco teórico permite tener un
conocimiento profundo del tema en estudio y que le da significado a la
investigación y/o desarrollo, de aquí se pueden generar nuevos conocimientos.
Recuerde, de las teorías que apoyen su proyecto dependerá la validez interna
y externa. 

\section{Antecedentes o estado del arte}  

Se revisan publicaciones recientes para ver si se encuentra algún proyecto
hecho en otra parte o en otro momento que se parezca a la que se esta
desarrollando y así examinar sus resultados y forma de enfocarla, de manera
que no se cometan los mismos errores, o simplemente aprovechar de ellos lo
que le sirva y oriente en el desarrollo del proyecto.

\section{Marco legal}  
\lipsum[1]