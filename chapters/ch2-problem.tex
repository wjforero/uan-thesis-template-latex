\chapter{Planteamiento del problema}

\section{Descripción del problema}

Describa la situación actual del problema, las características, los hechos y los
acontecimientos que lo enmarcan.
En síntesis, defina claramente por qué lo considera problema.

\section{Formulación del problema}

Dependiendo del tipo de proyecto, se debe realizar el planteamiento del
problema de investigación o desarrollo y/o innovación en términos de
necesidades y pertinencia. En el caso de proyectos de investigación, cuando el
problema lo amerite, puede formularse de forma clara una pregunta concreta
de investigación teniendo en cuenta que esta debe ser viable y pueda darse
respuesta al terminar el proyecto. Cuando no sea pertinente formular una
pregunta de investigación, el planteamiento del problema se realizará
mediante una hipótesis. Para los proyectos de desarrollo y/o innovación es
fundamental formular claramente los aspectos de innovación que se desea
resolver, en el contexto del problema a cuya solución o entendimiento se
contribuirá con la ejecución del proyecto. En todos los casos, se recomienda
además, hacer una descripción precisa y completa de la naturaleza y magnitud
del problema en función del desarrollo de la entidad, la institución o de su
pertinencia para el área de investigación.

\section{Justificación}  

Aquí se demuestra lo importante que es el desarrollo del proyecto, se debe
explicar claramente el motivo del porque y para qué se va a realizar el proyecto.
La justificación se realizará una vez se haya planteado el problema y establecido
los objetivos. En resumen, deberá preguntarse: ¿Por qué es importante el
problema planteado? ¿Cuál es su relevancia social, económica y profesional?.
Es importante tener presente cuál será el aporte que hará la investigación o
desarrollo.

\section{Objetivos}

Los objetivos deben mostrar una relación clara y consistente con la descripción
del problema y, específicamente, con el problema que se quiere resolver.
En general, los objetivos son acciones realizables, se determina si se va a
analizar, determinar, comprar, describir o interpretar.

\subsection{Objetivo general}  

En el Objetivo General, se determina el fin último del proyecto con respecto al
conocimiento sobre el objeto de estudio, esto quiere decir que se visualiza de
forma global el problema a trabajar. Se plantea en la forma: Que, como, para
que.

Se recomienda formular un solo objetivo general global, coherente con el
problema planteado, y más objetivos específicos que conducirán a lograr el
objetivo general y que son alcanzables con la metodología propuesta.

Redactarse los objetivo con un verbo en infinitivo al principio, este deberá
denotar la búsqueda de un conocimiento, por ejemplo: determinar, evaluar,
analizar, describir, desarrollar, descubrir, clasificar, enumerar, establecer,
experimentar, observar, obtener, proponer, comparar, intuir, percibir, 
capturar, acopiar, desarrollar, discutir, elaborar, recolectar, concentrar,
comprobar, aplicar, probar, inferir, aclarar, acoger, actualizar, abatir, adecuar,
debatir, afirmar, advertir, afrontar, agotar, ahondar, generar, guiar, comentar,
estudiar, estructura, reforzar, etc.

\subsection{Objetivos específicos}  

Estos son las metas a corto plazo para alcanzar el objetivo general.
Los objetivos específicos, señalan las actividades que se deben cumplir para
avanzar en la investigación o desarrollo y lo que se pretende lograr en cada
una de las etapas de ella, por ende, la suma de los resultados de cada uno de
los objetivos específicos integran el resultado del proyecto.

\begin{itemize}
    \item Los objetivos específicos se desprenden lógica y temáticamente del general.
    \item No pueden abarcar más que el objetivo general ni apuntar a propósitos
    diferentes o antagónicos del mismo.
    \item Los objetivos están relacionados directamente con el problema y su función
    dentro del proyecto.
    \item No confunda objetivos con actividades o procedimientos metodológicos.
    \item Recuerde que los objetivos deben plantearse de tal manera que sean medibles.
\end{itemize}

\section{Alcance y limitaciones del proyecto}
\lipsum[1]

\subsection{Alcance}  
\lipsum[1]

\subsection{Limitaciones}  
\lipsum[1]